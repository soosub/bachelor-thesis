\documentclass[]{article}

\usepackage{url}
\usepackage{hyperref}

%opening
\title{Quantum Adiabatic Algorithm and Trotterization}
\author{J. Bus}

\begin{document}
\maketitle

\section{Quantum Adiabatic Algorithm}
Have a look at the following resources
\begin{itemize}
	\item \href{https://arxiv.org/abs/quant-ph/0001106}{Paper} by Farhi, Goldstone, Gutmann, and Sisper
	\item \href{https://www.scottaaronson.com/qclec/25.pdf}{Lecture} on Hamiltonians by Aaronson
	\item \href{https://www.scottaaronson.com/qclec/26.pdf}{Lecture} on the Adiabatic Algorithm by Aaronson 
\end{itemize}

\section{Trotterization}
From \url{https://www.scottaaronson.com/qclec/25.pdf}, slightly adapted\\

Once adding Hamiltonians, one faces a mathematical question, namely: If $A$ and $B$ are matrices, is it generally the case that $e^{A+B} = e^Ae^B$ ? The answer is \textbf{no}. In the special case that $A$ and $B$ commute however, we find that the equality holds. What to do when they don't? Fortunately, there's a special trick for this, known as \textbf{Trotterization}. It uses the following approximation
\begin{equation}
	e^{A+B} \approx e^{\epsilon A}e^{\epsilon B}e^{\epsilon A}e^{\epsilon B}\dots e^{\epsilon A}e^{\epsilon B}
\end{equation}
each step $e^{\epsilon A}e^{\epsilon B}$ is repeated $1/\epsilon$ times. This basically means that we can achieve the same effect as $A$ and $B$ occurring simultaneously, by repeatedly switching between doing a tiny bit of $A$ and a tiny bit of $B$. We won’t do it here, but it’s possible to prove that the approximation improves as $\epsilon$ decreases, becoming an exact equality in the limit $\epsilon \to \infty$


\end{document}
