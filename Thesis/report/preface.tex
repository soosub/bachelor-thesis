\begin{abstract}
	The Quantum Approximate Optimization Algorithm (QAOA) is one of the promising near-term algorithms designed to find approximate solutions for combinatorial optimization problems. The algorithm prepares a parametrized state that is aimed to maximize the expectation value of the objective function of the problem. The circuit for QAOA consists of $p$ layers, and it depends on $2p$ parameters, the determination of which can be carried out using a variational quantum eigensolver (VQE) subroutine. This is a variational approach for finding optimal parameters, where the quantum state is prepared with a quantum computer, and a classical computer is used for calculating the corresponding objective function value and performing an outer loop optimization routine in order to find optimal parameters. Earlier work on the algorithm has shown that QAOA is closely related to the Quantum Adiabatic Algorithm, which results in noticeable patterns in the optimal parameters. To this purpose, two methods, INTERP and FOURIER, were proposed to exploit this structure by Zhou et al. \cite{ZWCPL18}. In this thesis I extend upon these results by examining the effectiveness of one of their methods, the INTERP method, on the Max-Cut problem from graph theory. The performance of QAOA using this method was studied on several classes of graphs and benchmarked against the currently best-known classical algorithm Goemans-Williamson, which achieves an approximation ratio of $\rho \approx 0.878$ on arbitrary graphs. The classes of graphs examined include cyclic, weighted and unweighted 3-regular, and Erd\H{o}s-R\'enyi graphs with edge probabilities of 0.5 and 0.75 and graph sizes ranging from 4 to 16. Moreover, it was investigated if the INTERP method offers an advantage compared to random initialization of parameters and whether or not the method is polynomial in $p$. It was found that different graph classes lead to similar but different parameter patterns. Furthermore, the performance of QAOA using the INTERP method was dependent on the graph class considered. For the small graphs investigated it was found that QAOA outperforms Goemans-Williamson on ER-0.50, ER-0.75 and weighted 3-regular graphs from $p=7$ and beyond with respect to how well the expectation of the objective function approximates the optimal value, while for unweighted 3-regular graphs $p=6$ suffices. However, the calculation of the parameters is costly and requires a number of function evaluations that is on the order of the number of possible bipartitions of small graph with sizes $n \lesssim 16$. Moreover, the numerical results suggested that indeed the method is polynomial in $p$ advocating that the algorithm might be a strong alternative to Goemans-Williamson for large graph sizes, once hardware allows it.
\end{abstract}



