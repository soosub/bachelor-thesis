\chapter{Introduction}
Quantum computing is a radically different kind of computing that makes use of the quantum mechanical nature of matter in order to process information. The two prominent applications of quantum computing you will often read or hear about are Shor's 1994 algorithm \cite{Shor94} and Grover's 1996 algorithm \cite{Grover96}, for prime factorization and search problems respectively. However, in order to actually run these algorithms a relatively large number qubits is required, on top of that, fault tolerant quantum computing is needed, requiring even more qubits. As we are now entering the era of Noisy Intermediate-Scale Quantum (NISQ) technology, we are unable to achieve these requirements at the moment. For this reason, we are interested in finding applications that \emph{are} viable to execute on the devices that are presently available, or will be in the near future \cite{Preskill18}. In particular, we want to know if there are useful algorithms that can be run on the quantum computers that are available in the short term. Hopefully, by exploring the possibilities, we will find new quantum algorithms that exhibit quantum advantage or even quantum supremacy \cite{QuantumSupremacy}, meaning they offer a speed-up or have capabilities beyond what is tractable on classical computers respectively. 

One of the promising algorithms that can be implemented on NISQ devices is the Quantum Approximate Optimization Algorithm, abbreviated QAOA, that was proposed by Farhi, Goldstone and Gutmann in 2014 \cite{FGG14}. The QAOA is an algorithm designed to find approximate solutions for combinatorial optimization problems. It is a hybrid quantum algorithm that makes use of both a classical processor as well as a quantum processor. This approach can be advantageous, especially in NISQ devices, as the need for long coherence times is avoided by only executing short calculations on a quantum computer. 

The algorithm prepares a parametrized state after which the state is sampled to infer an approximate solution for the problem. An obstacle for feasible use of the algorithm is the need to find good parameters. Another topic of interest is the performance of QAOA beyond its lowest depth variant, as it is largely unknown.

In this thesis I will explore how QAOA applied to the NP-complete Max-Cut problem compares to currently known classical algorithms, in particular the Goemans-Williamson algorithm \cite{GW95}, the best known classical approximation algorithm. This will be done by using a heuristic technique proposed in \cite{ZWCPL18}, called INTERP, for finding suitable, quasi-optimal parameters efficiently. In this work I will expand upon the results of \cite{ZWCPL18} by analyzing the performance of INTERP on different classes of graphs, namely cyclic graphs, 3-regular weighted and unweighted graphs, and Erd\H{o}s-R\'enyi graphs with edge probability 0.5 and 0.75. Moreover, the time complexity of the algorithm will be investigated using numerical results of the parameter optimization runs.

Throughout this work I will assume basic knowledge of linear algebra and that the reader is comfortable with terminology from quantum mechanics and quantum computing, such as qubit, superposition, interference and entanglement. For the interested reader without the necessary prerequisites I recommend the canonical \emph{Quantum Computation and Quantum Information} by Michael A. Nielsen and Isaac L. Chuang \cite{Mike&Ike}. For a more gentle and accessible introduction I can highly recommend the website \href{https://quantum.country/qcvc}{\emph{Quantum Country}} \cite{QuantumCountry} by Nielsen and Andy Matuschak. Other good resources I enjoyed reading are \cite{Hidary} and \cite{Qiskit-Textbook}.
\\

My thesis will be arranged as follows. I will start with a discussion of combinatorial optimization in Chapter \ref{chap:combop}, with emphasis on the Max-Cut problem. This will be followed by an overview of QAOA, and previous work done on the topic in Chapter \ref{chap:qaoa}. In Chapter \ref{chap:implementation} I will present a detailed discussion of the implementation of QAOA proposed in \cite{ZWCPL18}, which is used to produce the results presented in Chapter \ref{chap:results}. I will finalize this thesis with my conclusions and recommendations for future work in Chapter \ref{chap:conclusions}. 
